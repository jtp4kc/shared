\documentclass[]{article}

\usepackage[utf8]{inputenc}

\usepackage{geometry} % to change the page dimensions
\geometry{letterpaper} % letterpaper, a5paper, etc.
\geometry{margin=1in} % change the size of the margins

\usepackage{graphicx} % supports \includegraphics command
%\usepackage{wrapfig}
\graphicspath{ {images/} }

%\usepackage{parskip}
%\setlength{\parindent}{15pt}

\usepackage{booktabs} % better looking tables
\usepackage{array} % better arrays in math
\usepackage{paralist} % flexible lists
\usepackage{verbatim} % can add a comment block, better verbatim mode
\usepackage{subfig} % can include more than one figure per figure section
\usepackage[numbers,super,sort&compress]{natbib}
%\usepackage[ampersand]{easylist}
\usepackage[usenames,dvipsnames,svgnames,table]{xcolor}
\usepackage{setspace}
\usepackage{url}

%opening
\title{Multistate Bennett Acceptance Ratio}
\author{Tyler P.}
%\date{}

%%% Headers and footers
\usepackage{fancyhdr} % needs to be after 'geometry'
\pagestyle{fancy} % options: empty, plain, fancy
\renewcommand{\headrulewidth}{0pt} % can customize
\lhead{}\chead{}\rhead{}
\lfoot{}\cfoot{\thepage}\rfoot{}

%%% Section title appearance
%\usepackage{sectsty}
%\allsectionsfont{\sffamily\mdseries\upshape}

%%% ToC appearance
%\usepackage[nottoc,notlof,notlot]{tocbibind} % put bibliography in ToC
%\usepackage[titles,subfigure]{tocloft}
%\renewcommand{\cftsecfont}{\rmfamily\mdseries\upshape}
%\renewcommand{\cftsecpagefont}{\rmfamily\mdseries\upshape} % No bold!


\begin{document}
	
\maketitle

\begin{abstract}
	In order to better understand the re-weighting performed in MBAR free energy estimations, the math behind the method is re-derived here.
\end{abstract}

\section*{Definitions}
\textbf{Partition function}

$NVT =$ isometric, isothermal ensemble at fixed number of particles

$Z_k =$ canonical partition function (normalization constant)

$Z_n(x) =$ contribution to canonical partition function from one configuration

$P_{n,k}(x) =$ probability of the $n$-th microstate occuring in state $k$

$ A_k =$ (Helmholtz) free energy at a given $NVT$ state

$ F_k = \beta A_k$, the reduced free energy

$k_B$ is the Boltzmann constant, $T$ is the thermodynamic temperature, and $\beta = k_B T$

\begin{minipage}[b]{0.5\linewidth}
	\[ Z_n(x) = \exp(-\beta E_k(x)) \]
	\[ Z_k = \sum_{n=1}^{N} Z_n(x) \]
\end{minipage}
\begin{minipage}[b]{0.5\linewidth}
	\[ P_{n,k}(x) = \frac{Z_n(x)}{Z_k} \]
	\[ A_k = -k_B T \ln(Z_k) \]
\end{minipage}

where $x$ is a configuration and $n$ is its index,
and $i$ is a state and $k$ is its index

~\linebreak
Note that if two configurations $x_n$ have the same energy $E_k(x_n)$, then the same value of $Z_n$ will occur twice, and provide the proper multiplicity for that energy.

~\linebreak
\textbf{Metropolis Monte Carlo}\cite{bennett}

$MC$ is short for Monte Carlo, a process by which an `event' is allowed to occur only a certain percentage of the time, where that percentage is evaluated by the algorithm in question

$M(x) =$ Metropolis function, where $x$ is an observable

$\vec{q} = {<}q_1, q_2, ... q_N{>} =$ configurational degrees of freedom

$\Phi_x =$ configurational potential energy, and $K_x =$ momenta kinetic energy

$U_k(\vec{q}) = (E(x) - K_x)/k_B T = \beta \Phi_x$, reduced potential energy, as a function of configuration space

$Q_k = \int \exp(-U_k(\vec{q}))~d\vec{q}$, potential energy portion of the canonical partition function

\[ M(x) = \min \{ 1, \exp(-x) \} \]
\[ M(-x) = \min \{ 1, \exp(x) \} = 1/ \max \{ 1, \exp(-x) \} \]
\[ \frac{M(x)}{M(-x)} = \min \{ 1, \exp(-x) \} \max \{ 1, \exp(-x) \}  = \exp(-x) \]

because one of either $\min$ or $\max$ will always yield $1$, and the other will resultingly give $\exp(-x)$.

~\linebreak
Bennett uses this property of the Metropolis function to relate it to canonical potential functions (and their distributions $\exp(-U)$).

\[ \frac{M(U_1 - U_0)}{M(U_0 - U_1)} = \exp(U_0 - U_1) \]

~\linebreak
\textbf{Bennett Acceptance Ratio}\cite{bennett}

$f(x) = (1 + exp(x))^{-1}$, the Fermi distribution function

\section*{Derivation}

By rearrangement of the canonical variables, we can show that the probability of a microstate\cite{wikiCPF} is given as
\begin{minipage}[b]{0.32\linewidth}
	\[ A_k = -k_B T \ln(Z_k) \]
\end{minipage}
\hfill
\begin{minipage}[b]{0.32\linewidth}
	\[ \ln(Z_k) = -\frac{A_k}{k_B T} = -F_k \]
\end{minipage}
\hfill
\begin{minipage}[b]{0.32\linewidth}
	\[ Z_k = \exp \left( \frac{-A_k}{k_B T} \right) \]
\end{minipage}
\[ P_{n,k} = \frac{Z_n(x)}{\exp (-A_k/k_B T)} = \frac{\exp(-\beta E_k(x))}{\exp (-A_k/k_B T)} = \exp\left(\frac{-E_k(x)}{k_B T}\right) \exp \left( \frac{A_k}{k_B T} \right) = \exp\left(\frac{A_k-E_k(x)}{k_B T} \right), \]
such that the canonical partition function itself is no longer in the expression. Since probabilities must sum to one when normalized, we gain the following definition:
\[ 1 = \sum_{n=1}^{N} P_{n,k} = \sum_{n=1}^{N} \exp(\beta A_k - \beta E_k(x)) = \exp(\beta A_k) \sum_{n=1}^{N} \exp(- \beta E_k(x)) \]
\[ \exp(-\beta A_k) = \sum_{n=1}^{N} \exp(- \beta E_k(x)). \]
\[ A_k = -\frac{1}{\beta} \ln \left( \sum_{n=1}^{N} \exp(- \beta E_k(x)) \right), \]

which is consistent with direct substitution of the canonical variables into the free energy expression.

~\linebreak
Using the property of Metropolis functions, Bennett constructs the following line of relations\cite{bennett}:

\[ M(U_1 - U_0)~\exp(-U_0) = M(U_0 - U_1)~\exp(-U_1) \]
\[ \frac{Q_0}{Q_0} M(U_1 - U_0)~\exp(-U_0) = M(U_0 - U_1)~\exp(-U_1) \frac{Q_1}{Q_1} \]
\[ \int Q_0  \frac{M(U_1 - U_0)~\exp(-U_0)}{Q_0} d\vec{q} ~ = \int Q_1 \frac{M(U_0 - U_1)~\exp(-U_1)}{Q_1} d\vec{q} \]
\[ Q_0~\langle M(U_1 - U_0) \rangle_0 = Q_1~\langle M(U_0 - U_1) \rangle_1 \]
\[ \frac{Q_0}{Q_1} = \frac{\langle M(U_0 - U_1) \rangle_1}{\langle M(U_1 - U_0) \rangle_0} \]

or, more generally,
\[ \frac{Q_0}{Q_1} = \frac{Q_0 \int W \exp(-U_1 - U_0) d\vec{q}}{Q_1 \int W \exp(-U_0 - U_1) d\vec{q}} = \frac{\langle W \exp(-U_0) \rangle_1}{\langle W \exp(-U_1) \rangle_0} \]

where $W$ is a weighting function.

~\linebreak
Here we address the free energy difference, which is found as

\begin{minipage}[b]{0.45\linewidth}
	\[ F_0 = - \ln(Q_0 Q_{KE} ) \]
\end{minipage}
\hfill
\begin{minipage}[b]{0.45\linewidth}
	\[ F_1 = - \ln(Q_1 Q_{KE} ) \]
\end{minipage}

\[ \Delta F_{est} = F_1 - F_0 = \ln \left( \frac{Q_0 Q_{KE}}{Q_1 Q_{KE}} \right) = \ln \left( \frac{Q_0}{Q_1} \right). \]

We assume that the kinetic partition functions are equal and therefore cancel.

\begin{thebibliography}{9}
	\bibitem{bennett}
		Bennett, C. H.,
		(1976),
		Efficient Estimation of Free Energy differences from Monte Carlo Data.,
		\emph{J. Comput. Phys.},
		22, 
		245–268.
	\bibitem{wikiCPF}
		Wikipedia,
		Canonical ensemble --- Wikipedia{,} The Free Encyclopedia,
		2015,
		\url{http://en.wikipedia.org/w/index.php?title=Canonical_ensemble&oldid=659866755},
		[Online; accessed 3-June-2015]
\end{thebibliography}

\end{document}